\documentclass[11pt]{article}
\usepackage{amsmath}
\usepackage{amsfonts}
\usepackage{amssymb}
\usepackage{booktabs}
\usepackage{geometry}
\usepackage[colorlinks=true, linkcolor=blue, citecolor=blue, urlcolor=blue]{hyperref}
\usepackage{graphicx}
\geometry{margin=1in}

\title{Flash-SD-KDE: Accelerating SD-KDE with Tensor Cores}
\author{}
\date{}

\begin{document}
\maketitle

\section{Introduction}

Score-debiased kernel density estimation (SD-KDE)~\cite{epstein2025sdkde}
improves the statistical efficiency of standard kernel density estimators:
for sufficiently smooth densities, SD-KDE attains better bias and mean-squared
error rates than vanilla KDE while retaining a simple, nonparametric form.
These accuracy gains come with a significant computational drawback: the
empirical score used for debiasing introduces an additional $O(n^2)$ pass
over the data, so a naive implementation has roughly the same quadratic cost
as KDE itself but with a larger constant factor.  At realistic sample sizes,
this puts strong pressure on both the algorithmic structure and the hardware
implementation.

Given samples $x_i$ in one dimension and a bandwidth $h$, a Gaussian KDE is
\[
  \hat p(x)
  =
  \frac{1}{n h}
  \sum_{i=1}^n
  \varphi\!\left(\frac{x - x_i}{h}\right),
\]
and SD-KDE forms debiased samples
$x_i^{\mathrm{SD}} = x_i + \tfrac{h^2}{2}\,\hat s(x_i)$ where
$\hat s$ is an estimate of the score.  In this work we always use an
empirical score computed from the KDE itself, i.e.\ no parametric model
for the underlying density is assumed.  Writing the KDE explicitly in terms
of the samples,
\[
  \hat p(x)
  =
  \frac{1}{n h}
  \sum_{i=1}^n
  \varphi\!\left(\frac{x - x_i}{h}\right),
\]
we estimate the score as
\[
  \hat s(x)
  =
  \frac{\partial_x \hat p(x)}{\hat p(x)}
  =
  \frac{\sum_{i=1}^n \bigl[-\tfrac{x - x_i}{h}\,\varphi\!\left(\tfrac{x - x_i}{h}\right)\bigr]}
       {h^2 \sum_{i=1}^n \varphi\!\left(\tfrac{x - x_i}{h}\right)}.
\]

\section{Task}

The goal of Flash-SD-KDE is to implement a fast, GPU-accelerated SD-KDE
(with empirical score) on an NVIDIA RTX~A6000 using Triton.  We:
\begin{itemize}
  \item implement a Gaussian KDE and KDE-based score estimator in Triton,
  \item use these to construct an empirical SD-KDE on the GPU, and
  \item benchmark against CPU Silverman KDE and scikit-learn's KDE over
        a range of sample sizes $(n_{\text{train}}, n_{\text{test}})$.
\end{itemize}

\subsection{Hardware}

All experiments run on a workstation equipped with an NVIDIA RTX~A6000
(GA102).  This GPU exposes 84 streaming multiprocessors (SMs); each SM
contains 128 FP32 ALUs and 16 special function units (SFUs).  Because the
ratio of FP32 ALUs to SFUs is $128{:}16$, we treat one $\exp$ (issued on an
SFU) as costing the equivalent of $128/16 = 8$ FP32 flops in our models.
The card delivers roughly $40$~TFLOP/s of peak FP32 throughput and
approximately $770$~GB/s of GDDR6 bandwidth.

The host CPU is a dual-socket AMD EPYC~7763 system (``Milan'') configured
with $2 \times 64$ cores and two hardware threads per core (256 logical
CPUs reported by \texttt{lscpu}).  The processors boost up to
3.53~GHz, expose 512~MiB of shared L3 cache across the sockets, and provide
modern ISA extensions (AVX/AVX2, FMA, BMI1/2, SHA, VAES, etc.).  Unless
otherwise noted, all CPU benchmarks in this report run on this platform.

\section{High-dimensional SD-KDE and Tensor Cores}

The 1-D SD-KDE formulation provides useful intuition, but our primary goal is
to understand how the method scales in higher dimensions and how to exploit
Tensor Cores.  Let $x_i, y_j \in \mathbb{R}^d$ denote
training and query points, with bandwidth $h$ and squared Euclidean distance
\[
  \lVert x_i - y_j \rVert^2
  =
  \lVert x_i \rVert^2
  +
  \lVert y_j \rVert^2
  - 2\,x_i^\top y_j.
\]
Stacking training samples into $X \in \mathbb{R}^{n_{\text{train}}\times d}$
and queries into $Y \in \mathbb{R}^{n_{\text{test}}\times d}$, the
pairwise dot-product matrix
\[
  G
  =
  X Y^\top
  \in
  \mathbb{R}^{n_{\text{train}}\times n_{\text{test}}}
\]
dominates the arithmetic once $d$ is moderately large ($d\gtrsim 16$).  On
modern NVIDIA GPUs this GEMM can be mapped to Tensor Cores and evaluated at
5--10$\times$ the throughput of standard FP32 SIMT arithmetic, particularly
when we restrict to $d$ that are multiples of 16 and use Triton’s
\texttt{tl.dot} interface.  The remaining operations---vector norms,
broadcasted additions, and exponentials---are all $O(n_{\text{train}}
n_{\text{test}})$ but quickly become a small fraction of the total FLOPs as
$d$ grows.

The empirical SD-KDE score inherits the same GEMM structure.  Its numerator
involves terms of the form
\[
  \sum_j -(x_i - y_j)\,\varphi_{ij},
  \qquad
  \varphi_{ij}
  = \exp\!\left(-\frac{\lVert x_i - y_j\rVert^2}{2h^2}\right),
\]
which naively suggests $O(n_{\text{train}} n_{\text{test}} d)$ additional
elementwise arithmetic.  Using the identity
\[
  \sum_j (x_i - y_j)\,\varphi_{ij}
  =
  x_i \sum_j \varphi_{ij}
  -
  \sum_j \varphi_{ij}\,y_j,
\]
we can decompose the numerator into a second GEMM:
\[
  T
  =
  \Phi Y.
\]
Thus both the KDE evaluation and the SD-KDE score numerator reduce to
Tensor-Core-accelerable matrix multiplies, plus $O(n^2)$ scalar work for the
norms and exponentials.  In this note we focus on the $d=16$ case, which
aligns naturally with the Tensor Core tile sizes on the RTX~A6000 and offers a
clean contrast to the 1-D baseline (summarized in the appendix).

\subsection{Arithmetic intensity in $d$ dimensions}

To understand when the high-dimensional SD-KDE becomes compute-bound, we
estimate FLOPs, bytes moved, and arithmetic intensity for the $d$-dimensional
case.  Let $n_{\text{train}} = k$ and set $n_{\text{test}} = k/8$ as in the
experiments.

\paragraph{Total FLOPs.}
The 16-D implementation consists of three main matrix-multiply stages:
\begin{enumerate}
  \item Score Gram matrix $G = X X^\top$:
        $2 d k^2$ FLOPs.
  \item Score numerator $T = \Phi X$:
        $2 d k^2$ FLOPs, plus $4 k^2$ scalar FLOPs for norms and distance
        terms and $8 k^2$ FLOPs for exponentials (counting each $\exp$ as
        $8$ FLOPs due to the SFU/FP32 ratio).
  \item Final KDE Gram matrix on debiased data:
        $2 d k (k/8)$ FLOPs, plus $4 k (k/8)$ scalar FLOPs and
        $8 k (k/8)$ FLOPs for exponentials.
\end{enumerate}
Aggregating these terms yields
\[
  \text{FLOPs}_d(k)
  \approx
  4 d k^2 + 12 k^2 + 2 d \frac{k^2}{8} + 12 \frac{k^2}{8}
  =
  \Bigl(4 d + 12 + \tfrac{d}{4} + \tfrac{3}{2}\Bigr) k^2.
\]
Substituting $d=16$ gives
\[
  \text{FLOPs}_{16}(k)
  \approx
  81.5\,k^2,
\]
which is on the order of $10^{11}$ FLOPs for $k=32\text{k}$.

\paragraph{Bytes moved.}
In the idealized case where each sample and query is read once and each
output is written once, the leading-order memory traffic is
\[
  \text{Bytes}_d(k)
  \approx
  4\bigl(d k + d (k/8) + d k + k/8\bigr)
  =
  4\Bigl(\tfrac{9}{8} d k + \tfrac{k}{8}\Bigr),
\]
where we count reads of the training data $X$, debiased data, queries $Y$,
and writes of the $k/8$ outputs, up to lower-order terms.

\paragraph{Arithmetic intensity.}
Dividing FLOPs by bytes gives
\[
  I_d(k)
  =
  \frac{\text{FLOPs}_d(k)}{\text{Bytes}_d(k)}
  \approx
  \frac{\bigl(4 d + 12 + \tfrac{d}{4} + \tfrac{3}{2}\bigr) k^2}
       {4\bigl(\tfrac{9}{8} d k + \tfrac{k}{8}\bigr)}
  \sim
  C(d)\,k
  \quad\text{for large }k,
\]
with
\[
  C(d)
  \approx
  \frac{4 d + 12 + d/4 + 3/2}{4\cdot (9 d/8)}
  =
  \frac{(17/4)\,d + 27/2}{9 d/2}.
\]
For $d=16$ this simplifies to
\[
  C(16) \approx 2.0,
  \qquad
  I_{16}(k) \approx 2.0\,k\ \text{flops/byte}.
\]
Thus, as in the 1-D setting, the arithmetic intensity grows linearly with
problem size; even moderate $k$ values place the 16-D SD-KDE well into the
compute-bound regime on the RTX~A6000.

\section{16-D experiments}

We now summarize the empirical behavior of the 16-D implementation on the
RTX~A6000.  For each $n_{\text{train}}$ in $\{4\text{k}, 8\text{k}, 16\text{k},
32\text{k}\}$ we set $n_{\text{test}} = n_{\text{train}}/8$, draw data from a
simple 16-D Gaussian mixture, and run three baselines: scikit-learn KDE,
Torch SD-KDE (GEMM-based), and Triton SD-KDE (Tensor-Core GEMMs for both KDE
and score).  Figure~\ref{fig:nd_runtime} shows that the Triton SD-KDE rapidly
outpaces both baselines as $n$ grows, while remaining numerically close to the
Torch SD-KDE reference.

We also measure utilization by combining the flop model above with the
measured runtimes.  As shown in Figure~\ref{fig:triton_sd_kde_nd_util}, the
Torch SD-KDE baseline reaches only a few percent of the A6000 Tensor Core
peak, whereas the Triton implementation climbs into the multi-digit range once
$n_{\text{train}}$ exceeds $8\text{k}$.  This confirms that the 16-D
Tensor-Core formulation is firmly compute-bound and that additional tuning
effort should focus on kernel fusion and occupancy rather than memory traffic.

\begin{figure}[t]
  \centering
  \includegraphics[width=0.8\linewidth]{../nd-runtime.pdf}
  \caption{Runtime comparison for 16-D KDE/SD-KDE across $n_{\text{train}}$
           up to $32{,}768$ ($n_{\text{test}} = n_{\text{train}}/8$).
           The Triton SD-KDE enjoys both the Tensor-Core acceleration
           (relative to sklearn) and the GEMM-based score computation
           (relative to the Torch baseline).}
  \label{fig:nd_runtime}
\end{figure}

\begin{figure}[t]
  \centering
  \includegraphics[width=0.8\linewidth]{../triton-sd-kde-nd-util.pdf}
  \caption{Utilization (percentage of RTX A6000 Tensor Core peak, taken as
           $155\,\text{TFLOP/s}$ FP32-equivalent) for the 16-D SD-KDE
           pipeline, using the flop estimate
           $4 d n^2 + (4+8) n^2 + 2 d m n + (4+8) m n$ with $d=16$ and
           $m = n/8$.  Bars are annotated with the observed runtime (ms).}
  \label{fig:triton_sd_kde_nd_util}
\end{figure}

\section{Performance tuning}

We performed a sweep over the Triton launch parameters to improve utilization
for the $n_{\text{train}} = 32\text{k}$ case (with $n_{\text{test}}=4\text{k}$).
Specifically we varied:
\begin{itemize}
  \item $BLOCK\_M \in \{32, 64, 128, 256\}$,
  \item $BLOCK\_N \in \{32, 64, 128, 256\}$,
  \item $num\_warps \in \{1, 2, 4, 8\}$,
  \item $num\_stages \in \{1, 2, 4\}$,
\end{itemize}
and selected the combination that minimized runtime.  For this workload we
found that $BLOCK\_M=64$, $BLOCK\_N=128$, $num\_warps=1$, and $num\_stages=2$
gave the best overall performance, increasing measured FLOP utilization by
more than $2\times$ relative to the initial settings.  The same sweep can be
repeated for different problem sizes if further tuning is needed.

\appendix

\section{1-D SD-KDE baseline}

For completeness we briefly summarize the 1-D SD-KDE arithmetic intensity and
empirical behavior.  With $n_{\text{train}} = k$ and $n_{\text{test}} = k/8$,
there are two steps:
\begin{enumerate}
  \item \textbf{Score + shift:} For each training point we compute
        $\hat s(x_i)$ from all $k$ points and then form
        $x_i^{\mathrm{SD}} = x_i + \tfrac{h^2}{2}\hat s(x_i)$.  This uses
        $O(k^2)$ pairwise kernel interactions.  With the RTX~A6000 hardware
        ratio ($128$ FP32 ALUs, $16$ SFUs per SM) we budget an $\exp$ as
        $8$ flop-equivalents.  The score accumulation requires one $\exp$
        and roughly eight additional arithmetic ops (subtraction, scaling,
        accumulation), yielding $c_1 \approx 16$ flops per
        $(\text{train},\text{train})$ pair.
  \item \textbf{KDE on debiased samples:} We then evaluate a standard Gaussian
        KDE at $k/8$ query points using the $k$ debiased samples, which uses
        $O(k^2/8)$ interactions.  Each pair requires one $\exp$ (8 flops) plus
        about six other operations (difference, square, scaling, accumulation),
        so we take $c_2 \approx 14$ flops per $(\text{train},\text{test})$ pair.
\end{enumerate}
The total work is therefore approximated by
\[
  \text{FLOPs}(k)
  \;\approx\;
  c_1 k^2 + c_2\,k\,(k/8)
  \;\approx\;
  16 k^2 + 14\frac{k^2}{8}
  \;=\;
  17.75\,k^2.
\]
For $k = 32\text{k}$ this is on the order of $2\times 10^{10}$ flops.  When we
fix $n_{\text{test}} = k/8$ we move approximately $5k$ bytes (counting one
read of each train and test point and one write of each output), so the
arithmetic intensity scales as
\[
  I(k)
  =
  \frac{\text{FLOPs}(k)}{\text{Bytes}(k)}
  \;\approx\;
  \frac{17.75\,k^2}{5\,k}
  \approx
  3.55\,k
  \quad \text{flops/byte},
\]
placing realistic problem sizes firmly in the compute-bound regime.

Empirically we sweep $k$ over powers of two from $512$ to $32\text{k}$, with
$n_{\text{test}} = k/8$, and average over three seeds.  For each
configuration we record CPU Silverman KDE time, scikit-learn Gaussian KDE
time, and SD-KDE GPU time.  Across the entire range, the empirical SD-KDE GPU
implementation is consistently faster than scikit-learn, with speedups
growing with $k$; at the largest sizes the GPU achieves well over an
order-of-magnitude speedup relative to the sklearn baseline.

\begin{figure}[t]
  \centering
  \includegraphics[width=0.8\linewidth]{../flash-sd-kde.pdf}
  \caption{Average runtime (log-scale $y$-axis) of scikit-learn KDE and
           SD-KDE GPU across $n_{\text{train}}$ in 1-D; annotations show
           the SD-KDE GPU speedup relative to sklearn.}
  \label{fig:flash_sd_kde}
\end{figure}

\begin{figure}[t]
  \centering
  \includegraphics[width=0.8\linewidth]{../emp-sd-kde-util.pdf}
  \caption{Estimated GPU utilization (as a percentage of A6000 FP32 peak)
           for 1-D SD-KDE implemented in Triton and optimized Torch,
           computed from the 1-D flop model and the measured runtimes.}
  \label{fig:flash_sd_kde_util}
\end{figure}

We also run a Triton-only sweep over larger 1-D problem sizes, using powers of
two for $n_{\text{train}}$ up to $2^{22} \approx 4.2$ million (and
$n_{\text{test}} = n_{\text{train}}/8$) with a single seed.  The helper
script \texttt{run\_triton\_scaling.sh} executes this experiment by invoking
the \texttt{--emp-kernel-only} benchmark mode, producing a compact log that we
feed to Nsight Systems when examining the multi-million-sample kernels.

\begin{figure}[t]
  \centering
  \includegraphics[width=0.8\linewidth]{../triton-large-util.pdf}
  \caption{Utilization of the 1-D Triton SD-KDE kernel for large
           $n_{\text{train}}$ (powers of two up to $2^{22}$).  Labels show
           the observed runtime for each data size; error bars are omitted
           because a single seed is used per point.}
  \label{fig:triton_large_util}
\end{figure}

\section{Additional commands}

For convenience we summarize the most frequently used scripts:
\begin{itemize}
  \item \texttt{run\_triton\_scaling.sh}: sweeps 1-D SD-KDE with \texttt{--emp-kernel-only} for Nsight profiling.
  \item \texttt{run\_triton\_sd\_kde\_nd.sh}: runs Triton-only 16-D SD-KDE sweeps and logs runtimes for utilization plots.
  \item \texttt{run\_nd\_runtime\_sweep.sh}: collects runtime data (sklearn vs Torch vs Triton) for 16-D KDE/SD-KDE up to $n=32{,}768$.
\end{itemize}

\bibliographystyle{plain}
\bibliography{references}

\end{document}
